\documentclass{article}

\usepackage{amsthm}
\usepackage{amssymb}
\usepackage{physics}
\usepackage{mathtools}
\usepackage{enumitem}
\usepackage{amsmath}
\usepackage[brazilian]{babel}

\newtheorem{axioma}{Axioma}
\newtheorem{conceitoPrimitivo}{Conceito Primitivo}

\newtheorem{prop}{Proposição}[section]
\theoremstyle{theorem}
\newtheorem{teo}{Teorema}
\theoremstyle{lemma}
\newtheorem{lema}{Lema}
\newtheorem{inferencia}{Regra de Inferência}
  
\usepackage[
    type={CC},
    modifier={by-nc-sa},
    version={3.0},
]{doclicense}

\usepackage{hyperref}
\hypersetup{
    colorlinks,
    citecolor=black,
    filecolor=black,
    linkcolor=black,
    urlcolor=black
}
\title{Prova 2}
\author{Tales da Silva Amaral}
\date{\today}

\theoremstyle{definition}
\newtheorem{definicao}{Definição}[section]

\theoremstyle{remark}
\newtheorem{obs}{Observação}[section]
\newtheorem{exemp}{Exemplo}[section]
\newtheorem{exercicio}{Exercício}[subsection]
\DeclareMathOperator{\card}{card}
\DeclareMathOperator{\Null}{Null}

\begin{document}
\maketitle
\newpage

\section{introdução}
\begin{definicao}[Transformação Bilinear]
	Uma transformação \( B : V\times W \to \mathbb{R}^m \) é 
\end{definicao}

\section{Questão 1}
\begin{prop}[Questão 1]
	Sejam \(f: U \subset \mathbb{R}^n \to \mathbb{R}^p \) e \( g: V \subset \mathbb{R}^p \to \mathbb{R}^m \) funções de classe \( C^2\)  nos abertos \( U\) e \(V\). Se \( x\in U\) e \( f(x) \in V\) , então \[ D^{2}(g\circ f)(x) = D^2g(f(x)) \circ \left(Df(x), Df(x)\right) + Dg(f(x)) \circ D^2f(x). \]
\end{prop}
\begin{proof}
	Dado \( x\in U \) com \( f(x) \in V\). Pela Regra da Cadeia, temos que \( D( g\circ f)(x) = Dg(f(x)) \circ Df(x) \). Tomando \(m(A,B) = A\circ B\), temos \(D(g\circ f)(x) = Dg(f(x))\circ Df(x) = m(Dg(f(x)), Df(x)) \). Temos que \( m\) é \(C^{\infty} \subset C^2 \) e \( f,g \subset C^2\), logo \( D(g\circ f) \) é diferenciável. Logo:
\begin{align*}
	D^2(g\circ f)(x) &= D( D(g\circ f)(x)) \\~\\
	&= D(m( Dg(f(x)), Df(x))) \\~\\
	&= Dm( Dg(f(x)), Df(x)) \circ D( Dg(f(x)), Df(x))  \\~\\
	&= Dm( Dg(f(x)), Df(x)) \circ ( D^2g(f(x)) \circ Df(x) , D^2f(x))  \\~\\
	&= m( Dg(f(x)) , D^{2}f(x) )  + m(D^{2}g(f(x)) \circ Df(x) , Df(x)) \\~\\
	&=  Dg(f(x)) \circ D^{2}f(x)   + D^{2}g(f(x)) \circ Df(x) \circ  Df(x) 
\end{align*}
\end{proof}
\section{A definir}
\begin{definicao}[$L(\mathbb{R}^n, \mathbb{R}^m)$]
	$$L(\mathbb{R}^n, \mathbb{R}^m) = \left\{ T \in F(\mathbb{R}^n, \mathbb{R}^m) \: | \: T \text{ é linear}\right\}$$
\end{definicao}
\begin{definicao}[$L(\mathbb{R}^n)$]
	$$L(\mathbb{R}^n) = L(\mathbb{R}^n, \mathbb{R}^n)$$
\end{definicao}
\begin{definicao}[$GL(\mathbb{R}^n)$]
	$$GL(\mathbb{R}^n) = \left\{ T \in L(\mathbb{R}^n) \: | \: T \text{ é bijetiva}\right\}$$
\end{definicao}
\begin{prop}
	$GL(\mathbb{R}^n)$ é aberto.
\end{prop}
\begin{proof}
	Seja $T\in GL(\mathbb{R}^n)$. Logo $T$ é  invertível (bijetiva). Tomando \( \delta = \dfrac{1}{|T^{-1}|} \). Supondo \( S \in B( T, \delta) \subset \mathcal{L}( \mathbb{R}^n) \). Dado \( x\in \mathbb{R}^n \setminus \{0\} \), temos 
\begin{align*}
 \delta |x| &= \delta | T^{-1}\circ T (x)| \\~\\
&\leq \delta |T^{-1}|\cdot |T(x)| \\~\\
&= |T(x)|\\~\\
&= |T(x) - S(x) + S(x) | \\~\\
&\leq |T(x) - S(x)| + |S(x) | \\~\\
&= |(T- S)(x)| + |S(x) | \\~\\
&\leq |(T- S)|\cdot |x| + |S(x) | \\~\\
&< \delta  |x| + |S(x) | 
\end{align*}

Logo \( |S(x)| + \delta |x| >  \delta|x| \implies |S(x)| > 0 \implies S(x) \neq 0 \). Como \( x\neq 0 \implies S(x) \neq 0 \), temos que \( S\) é injetiva. Como \( S \in \mathcal{L}(\mathbb{R}^n ) \), temos que \( S\) é um isomorfismo. Logo \( B(T,\delta ) \subset GL(\mathbb{R}^n ) \).
\end{proof}
\begin{prop}
	$f:GL(\mathbb{R}^n) \to GL(\mathbb{R}^n)$, dada for $f(T) = T^{-1}$ é contínua.
\end{prop}
\begin{proof}
	%carro
\end{proof}
\begin{prop}
Seja $f: GL(\mathbb{R}^n) \to GL(\mathbb{R}^n)$, dada por $f(T) = T^{-1}$.  Temos $f$ diferenciável.
\end{prop}
	\begin{proof}
	Temos  \begin{align*}
(T+H)(T+H)^{-1} = I &\iff \\~\\
T(T+H)^{-1} +H(T+H)^{-1} = I &\iff \\~\\
T(T+H)^{-1} = I -H(T+H)^{-1} &\iff \\~\\
(T+H)^{-1} = T^{-1}  ( I -H(T+H)^{-1} )  &\iff \\~\\
\end{align*}

Vou substituir $(T+H)^{-1}$ na equação acima.\begin{align*}
(T+H)^{-1} &= T^{-1}  ( I -H(T+H)^{-1} )   \\~\\
 &= T^{-1}  ( I -H \left[T^{-1}  ( I -H(T+H)^{-1} )\right])   \\~\\
 &= T^{-1}  ( I -H T^{-1}  ( I -H(T+H)^{-1} ))   \\~\\
 &= T^{-1}  ( I -H T^{-1}   +HT^{-1}H(T+H)^{-1} )   \\~\\
 &= T^{-1}   -T^{-1}H T^{-1}   +T^{-1}HT^{-1}H(T+H)^{-1}    \\~\\
\end{align*}

Se chamarmos $S_T(H) = -T^{-1}HT^{-1}$, temos \begin{align*}
f(T+H) &= (T+H)^{-1}    \\~\\
 &= T^{-1}   -T^{-1}H T^{-1}   +T^{-1}HT^{-1}H(T+H)^{-1}    \\~\\
 &= f(T)   +S_T(H)   +T^{-1}HT^{-1}H(T+H)^{-1}    \\~\\
\end{align*}

Afirmo que $S_T(H) = Df(T)(H)$. De fato, $S_T$ é linear (confia) e temos \begin{align*}
\displaystyle\lim_{H\to 0} \dfrac{| f(T+H) - f(T) - S_T(H)|}{|H|} &= \displaystyle\lim_{H\to 0}\dfrac{| +T^{-1}HT^{-1}H(T+H)^{-1} |}{|H|} \\~\\
&\leq \displaystyle\lim_{H \to 0}\dfrac{|T^{-1}|\cdot |H|\cdot |T^{-1}| \cdot |H| \cdot |(T+H)^{-1}|}{|H|} \\~\\
&= \|T^{-1}\|^2\cdot \displaystyle\lim_{H \to 0}|H| \cdot |(T+H)^{-1}| \\~\\
&=0
\end{align*}
	\end{proof}

\end{document}
