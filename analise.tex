\documentclass{article}

\usepackage{amsthm}
\usepackage{amssymb}
\usepackage{mathtools}
\usepackage{amsmath}
\theoremstyle{plain}
\newtheorem{axioma}{Axioma}
\newtheorem{prop}{Proposição}
\newtheorem{lema}{Lema}

\theoremstyle{definition}
\newtheorem{definicao}{Definição}[section]

\theoremstyle{remark}
\newtheorem{obs}{Observação}[section]
\newtheorem{exemp}{Exemplo}[section]

\begin{document}
\section{Teoria de conjuntos}
\section{Conjuntos Finitos e Infinitos}
\subsection{Números naturais}
\paragraph{Temos como conceitos primitivos o conjunto dos naturais, denotado por $\mathbb{N}$, cujos elementos são os números naturais, e uma função $s:\mathbb{N} \to \mathbb{N}$. Para cada $n\in \mathbb{N}$, o número $s(n)$ é o sucessor de $n$.  Temos os axiomas:}
\begin{axioma}
	$s:\mathbb{N} \to \mathbb{N}$ é injetiva.
\end{axioma}
\begin{axioma}
	$\mathbb{N} - s(n) = \left\{ 1 \right\}$. Ou seja, só existe um número natural que não é sucessor de nenhum outro, e ele é denotado por $1$.
\end{axioma}
\begin{prop}
	\label{propsuc}
	Todo natural diferente de $1$ possui um antecessor.
\end{prop}
\begin{proof}
	Seja $n\neq 1$ um número natural. Suponha que não exista $n_0$ natural com $s(n_0) = n$. Logo $n \not \in s\left(\mathbb{N}\right)$. Logo $n\in \mathbb{N} - s(n)$. Mas $\mathbb{N} - s(n) = \left\{ 1 \right\}$. Logo $n = 1$. Contradição. Logo existe  $n_0\in \mathbb{N}$ tal que $s(n_0) = n$.
\end{proof}
\begin{obs}
	Observe que a função $s: \mathbb{N} \to \mathbb{N}\setminus \{1\}$ é injetiva por definição e sobrejetiva pela proposicao \ref{propsuc}, logo é uma bijeção entre um subconjunto dos naturais com os naturais.
\end{obs}
\begin{axioma}[Princípio de indução]
	Se $X\subset \mathbb{N}$ é  um subconjunto tal que: 
	$$
	\begin{cases}
		1 \in X\\~\\
		n \in X \implies s(n) \in X
	\end{cases}
	$$
	Então $\mathbb{N} = X$.
\end{axioma}
\begin{definicao}[Soma]
	Dados $m,n\in \mathbb{N}$, sua soma $m+n$ é definida como:$$ m+n \coloneqq s^n(m).$$ A soma deve obedecer \begin{equation} m+1 = s(m)\end{equation} \begin{equation}m +s(n) = s(m+n)\end{equation}  para todos os $m,n$ naturais.
\end{definicao}
\begin{obs}
	Dedekind prova o "Teorema da Definição por Indução" para garantir que a notação $s^n(m)$ faça sentido.
\end{obs}
\begin{prop}[Associatividade da Soma]
	Para todos $p,m,n \in \mathbb{N}$, temos $m+(n+p) = (m+n) +p$.
\end{prop}
\begin{proof}
	Seja $X = \left\{ p \in \mathbb{N} \: | \forall m,n\in \mathbb{N} \: : \: m+(n+p) = (m+n) + p \right\}$. Da definição de adição, temos pra qualquer $m,n$ que $n+1 = s(n)$, logo $m+(n+1)  = m+s(n) = s(m+n) = (m+n) +1 \implies m+(n+1) = (m+n) +1 $. Logo $1\in X$.  Se $p \in X$, temos $m+(n+p) = (m+n) +p $. Logo \begin{align*} 
		m+(n+s(p)) &= m+ s(n+p) \\~\\
		&= s\left(m+(n+p)\right) \\~\\ 
		&= s\left((m+n)+p\right) \\~\\ 
		&= (m+n) + s(p).
	\end{align*}
	Logo $p \in X \implies s(p) \in X$. Temos que $X =  \mathbb{N}$ pelo princípio de indução. Logo a soma é associativa nos naturais.
\end{proof}
\begin{lema}[Comutatividade da soma com o $1$]
	\label{lema1}
	Para todo $m \in \mathbb{N}$, temos $m+1 = 1 +m$.
\end{lema}
\begin{proof}
	Seja $X = \left\{ m \in \mathbb{N} \: |  m+1 = 1+m \right\}$. Temos $1\in X$, pois $1+1 = 1+1$. Supondo $m\in X$, logo $m+1 = 1+m$. Temos \begin{align*} 
		1+ s(m)  &= s(1+m) \\~ \\
		&= s(m+1) \\~\\
		&= (m+1) +1 \\~\\
		&= s(m)+1
	\end{align*}
	Como $m\in X \implies s(m) \in X$ e $1\in X$, temos $X= \mathbb{N}$.
\end{proof}
\begin{prop}[Comutatividade da soma]
	Para todos $m,n \in \mathbb{N}$, temos $m+n = n +m$.
\end{prop}
\begin{proof}
	Seja $X = \left\{ m \in \mathbb{N} \: | \forall n \in \mathbb{N} \: : \:  m+n = n+m \right\}$. Temos $1\in X$  pelo  Lema \ref{lema1}.
	Supondo $m\in X$, logo $m+n = n+m$ para todo $n\in \mathbb{N}$. Temos \begin{align*}
		n + s(m) &= s(n+m)  \\~\\
		&= s(m+n) \\~\\
		&= (m+n) + 1\\~\\
		&= 1+(m +n )\\~\\
		&= (1+m )+n \\~\\
		&= (m+1 )+n \\~\\
		&= s(m) +n \\~\\
	\end{align*}
	Como $1\in X$ e $m \in X \implies s(m) \in X$, temos $X = \mathbb{N}$ pelo princípio de indução.
\end{proof}
\begin{prop}[Lei do corte]
	Para todos $m,n, p  \in \mathbb{N}$, temos $m+n = m +p \implies n = p $.
\end{prop}
\begin{proof}
	Seja $X = \left\{ m \in \mathbb{N} \: | \forall n\in \mathbb{N}\: \forall p \in \mathbb{N} \: : \:  m+n = m+p\implies n=p \right\}$. Temos $1\in X$  pois $1+n = 1+p \implies n+1 = p+1 \implies s(n) = s(p) \implies n =p$ pela injetividade de $s$.
	Supondo $m\in X$, temos $m+n = m+ p \implies n = p$ para todos $n,p$ naturais. Temos \begin{align*}
		s(m) + n = s(m)+p &\implies \\~\\
		n + s(m)  =  p + s(m) &\implies \\~\\
		s(n + m)  =  s(p + m) &\implies \\~\\
		n+m = p+m &\implies \\~\\
		m+ n = m+p &\implies \\~\\
		n = p.
	\end{align*}
	Logo $s(m) +n = s(m) + p \implies n = p$.
	Como $1\in X$ e $m \in X \implies s(m) \in X$, temos $X = \mathbb{N}$ pelo princípio de indução.
\end{proof}
\begin{lema}[Não existem ciclos nos naturais]
	\label{lema2}
	Para todos $m,p\in \mathbb{N}$, temos $m\neq m+p$.
\end{lema}
\begin{proof}
	Suponha que $m = m+p$ com $m,p\in \mathbb{N}$. Logo $s(m) = s(m+p) \implies m+1 = (m+p) +1 \implies m+1 = m+(p+1) \implies 1 = p+1 \implies s(p) = 1$. Como $1$ não é sucessor de nenhum natural, temos uma contradição. Logo $m\neq m+p$ para todos naturais $m,p$.
\end{proof}
\begin{lema}[Unicidade da Tricotomia]
	\label{lema3}
	Dados dois naturais $m$ e $n$, apenas uma das 3 possibilidades ocorre:
	$$
	\begin{cases}
		m = n \\~\\
		\exists p \in \mathbb{N} \: : \: m = n+p \\~\\
		\exists q \in \mathbb{N} \:  : \:n = m+q 
	\end{cases}
	$$
\end{lema}
\begin{proof}
	Pelo lema \ref{lema2}, se $m=n$, não podemos ter $m = n+p = m+p$ ou $n = m+q = n+q$ para algum $p,q \in\mathbb{N}$. Se $\exists p \in \mathbb{N} \: : \:m = n+p$, não podemos ter $m = n$ pelo lema \ref{lema2} e não podemos ter $\exists q \in \mathbb{N} \: : \:n = m+q$, pois  teríamos $m = n+p = (m+q)+p = m+(q+p) \implies  m = m +(q+p)$, que contradiz o lema \ref{lema2}.
\end{proof}
\begin{prop}[Tricotomia]
	\label{tricotomia}
	Dados dois naturais $m$ e $n$, exatamente uma das 3 possibilidades ocorre:
	$$
	\begin{cases}
		m = n \\~\\
		\exists p \in \mathbb{N} \: : \: m = n+p \\~\\
		\exists q \in \mathbb{N} \:  : \:n = m+q 
	\end{cases}
	$$
\end{prop}
\begin{proof}
	Seja $X = \left\{m \in \mathbb{N} | \forall n \in \mathbb{N} \: : \: (m = n ) \lor (	\exists p \in \mathbb{N} \: : \: m = n+p ) \lor ( 	\exists q \in \mathbb{N} \: : \: n = m+q ) \right\}$, ou seja: o conjunto dos números naturais que satisfazem pelo menos uma das condições da tricotomia para todo $n$.  

	$1\in X$, pois dado $n\in \mathbb{N}$, temos $n = 1$ ou $n\neq 1$. Se $n=1$, temos $m =1 = n$. Se $n\neq 1$, como $\mathbb{N} - s(\mathbb{N}) = \left\{1\right\}$, temos que existe um $n_0\in \mathbb{N}$ tal que $s(n_0) = n$. Logo $n = n_0 +1 \implies \exists q \: : \: n = q+1  = q+m $.
	
	Supondo $m\in X$. Dado $n\in \mathbb{N}$, se $m = n$, temos $s(m) = s(n) = n+1$, logo $\exists p \in\mathbb{N} \: : \: s(n) = n+p$. Se  $\exists p \in\mathbb{N} \: : \: m = n+p$, temos $s(m) = s(n+p) = (n+p+1) = n +s(p)$ , logo $\exists p' \in\mathbb{N} \: : \: s(n) = n+p'$. Se $\exists q \in \mathbb{N} \: : \: n = m+q $ com $q=1$, temos $n = m+1 = s(m)$. Se $\exists q \in \mathbb{N} \: : \: n = m+q $ com $q\neq 1$, existe $q_0\in \mathbb{N}$ tal que $s(q_0) = q$, logo temos $n = m+q = m +s(q_0) = m+(q_0+1) = m+1+q_0 = s(m)+q_0 \implies \exists q' \in \mathbb{N} \: : \: n = s(m)+q' $. 

	Como $1\in X$ e $m\in X \implies s(m) \in X$, temos $X = \mathbb{N}$.  Logo para todo par $m,n\in \mathbb{N}$, pelo menos uma das condições da tricotomia ocorre. Pelo lema \ref{lema3}, apenas uma das possbilidades ocorre. 
	%Melhorar
\end{proof}
\begin{definicao}[$<$]
	$$ m<n \iff \exists p\in \mathbb{N} \: : \: n = m+p$$
	Dados $m,n$ naturais, dizemos que $m$ é menor que $n$ ( $m<n$) quando existe $p\in \mathbb{N}$ tal que $n = m+p$.
\end{definicao}

\begin{prop}
	Temos $1 < n $ para todo $ 1 \neq n \in \mathbb{N}$ .
\end{prop}
\begin{proof}
	Como $n \neq 1$, temos pela proposição \label{propsuc} que $n$ possui um antecessor. Logo existe $n_0$ tal que $s(n_0) =n \implies n  = 1+n_0$. Logo $1<n$.
\end{proof}
\begin{definicao}[$\leq$]
	$$m \leq n \iff (m = n) \lor (m <n)$$
\end{definicao}
\begin{prop}[Transitividade da relação $<$]
	$m< n \land n< p \implies m < p$
\end{prop}
\begin{proof}
	Se $m<n$ e $n<p$, temos $n = m+q$ e $p = n+r$ para algum par $q,r\in \mathbb{N}$. Logo $p = n +r = (m +q) +r =m +(q+r)$. Logo $m<p$.
\end{proof}
\begin{prop}[Tricotomia da relação $<$]
	Dados $m,n\in \mathbb{N}$, exatamente uma das afirmações ocorre: $m=n$, ou $m<n$, ou $n<m$.
\end{prop}
\begin{proof}
	Segue diretamente da proposição \ref{tricotomia}.
\end{proof}
\begin{prop}
	Dados $m,n,p$ naturais, temos $$m +p < n+p \implies m <n.$$
\end{prop}
\begin{proof}
	Temos $m+p < n+p \implies \exists q \in \mathbb{N} \: : \: n+p = (m+p) + q \implies  \exists q \in \mathbb{N} \: : \: n = m + q \implies m<n$.
\end{proof}
\begin{definicao}[Multiplicação]
	Para todo $m\in \mathbb{N}$, seja $f_m : \mathbb{N} \to \mathbb{N}$ que associa cada $p\in \mathbb{N}$ a $f_m(p) = m+p$.
	Dados $m,n \in \mathbb{N}$, o produto entre naturais satisfaz $m\cdot 1 = m$ e  $ m\cdot (n+1) = (f_m)^n(m)$ .
\end{definicao}
\begin{lema}[ Distributiva do sucessor]
	$$m\cdot(n+1) = mn +m$$
\end{lema}
\begin{proof}
	\label{lemaDistributiva1}
	Se $n = 1$, temos  $m\cdot(1+1) = (f_m)^1(m) = f_m(m) = m+m = m\cdot 1 +m$. Se $n\neq 1$, existe $n_0\in \mathbb{N}$ tal que $s(n_0) = n$. Logo temos $m\cdot( n +1) = (f_m)^{n}(m) = (f_m)^{s(n_0)}(m) =   f_m( (f_m)^{n_0} (m)) = f_m( m(n_0+1)) = f_m(m \cdot n ) =  mn+m$.
\end{proof}
\begin{prop}[ Distributiva à esquerda]
	$$m\cdot(n+p) = mn +mp$$
\end{prop}
\begin{proof}
	Seja $X = \left\{p \in \mathbb{N} | \forall m,n \in \mathbb{N} \: : \:  n\cdot(m+p) = nm + np\right\}$. Temos $1\in X$ pelo lema \ref{lemaDistributiva1}. Supondo $p\in X$. Temos \begin{align*}
		n\cdot(m+s(p)) &= n\cdot ((m+p)+1)  \\~\\
		&= n\cdot(m+p) + n\\~\\
		&= nm+np+n\\~\\
		&=nm+n(p+1)\\~\\
		&=nm+n\cdot s(p)
	\end{align*}
	Como $p\in X\implies s(p) \in X $ e $1\in X$, temos $X = \mathbb{N}$.
\end{proof}
\begin{prop}[ Distributiva à direita]
	$$(m+n)\cdot p = mp +np$$
\end{prop}
\begin{proof}
	Seja $X = \left\{p \in \mathbb{N} | \forall m,n \in \mathbb{N} \: : \:  (m+n)\cdot p = mp + np\right\}$. Temos $1\in X$, pos $(m+n)\cdot 1 = m+n = m\cdot 1 + n \cdot 1$ . Supondo $p\in X$. Temos \begin{align*}
		(m+n)\cdot s(p) &= (m+n)\cdot(p+1)  \\~\\
		&= (m+n)\cdot p + (m+n)\\~\\
		&= mp+np+m+n\\~\\
		&=mp+m+ np+n\\~\\
		&=m(p+1)+ n(p+1)\\~\\
		&=m\cdot s(p) + n\cdot s(p)
	\end{align*}
	Como $p\in X\implies s(p) \in X $ e $1\in X$, temos $X = \mathbb{N}$.
\end{proof}
\begin{prop}[Associatividade]
	$$m\cdot(n\cdot p) = (m\cdot n ) \cdot p$$
\end{prop}
\begin{proof}
	Seja $X = \left\{p \in \mathbb{N} | \forall m,n \in \mathbb{N} \: : \:  m\cdot( n\cdot p) = (m\cdot n )\cdot p \right\}$. Temos $m\cdot (n\cdot1) = m\cdot n = (m\cdot n ) \cdot 1$, logo $1\in X$. \\ ~\\ Supondo $p\in X$. Temos 
	 \begin{align*}
		 m\cdot (n\cdot s(p)) &= m\cdot(n\cdot ( p+1) )  \\~\\
		 &= m\cdot ( n\cdot p + n) \\~\\
		 &= m\cdot ( n\cdot p) + m\cdot n \\~\\
		 &= (m\cdot  n)\cdot p + (m\cdot n) \\~\\
		 &= (m\cdot  n)\cdot (p+1) \\~\\
		 &= (m\cdot  n)\cdot s(p) \\~\\
	\end{align*}
	Como $p\in X\implies s(p) \in X $ e $1\in X$, temos $X = \mathbb{N}$.
\end{proof}
\begin{lema}[Comutatividade com $1$]
	\label{comutatividade1}
	$$m\cdot 1 = 1 \cdot m $$
\end{lema}
\begin{proof}
	Seja $X = \left\{m \in \mathbb{N} |  m\cdot 1 = 1\cdot m  \right\}$. Temos $1\cdot 1 = 1\cdot 1  $, logo $1\in X$. Supondo $m\in X$. Temos 
	 \begin{align*}
		 s(m)\cdot 1 &= (m+1)\cdot 1   \\~\\
		 &= m+1  \\~\\
		 &= m\cdot 1 + 1\cdot1  \\~\\
		 &= 1\cdot m +1\cdot 1   \\~\\
		 &=1\cdot (m+1)   \\~\\
		 &=1\cdot s(m)   \\~\\
	\end{align*}
	Como $m\in X\implies s(m) \in X $ e $1\in X$, temos $X = \mathbb{N}$.
\end{proof}
\begin{prop}[Comutatividade]
	$$m\cdot n = n \cdot m $$
\end{prop}
\begin{proof}
	Seja $X = \left\{n \in \mathbb{N} | \forall m \in \mathbb{N} \: : \:  m\cdot n = n\cdot m  \right\}$. Temos $1\in X$ pelo lema \ref{comutatividade1}. Supondo $n\in X$. Temos 
	 \begin{align*}
		 m\cdot s(n) &= m\cdot(n +1 )  \\~\\
		 &=mn + m\cdot1 \\~\\
		 &=nm + 1\cdot m \\~\\
		 &=(n+1)\cdot m  \\~\\
		 &=s(n)\cdot m \\~\\
	\end{align*}
	Como $p\in X\implies s(p) \in X $ e $1\in X$, temos $X = \mathbb{N}$.
\end{proof}
\end{document}
