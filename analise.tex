\documentclass{article}

\usepackage{amsthm}
\usepackage{amssymb}
\usepackage{mathtools}
\usepackage{amsmath}
\theoremstyle{plain}
\newtheorem{axioma}{Axioma}
\newtheorem{prop}{Proposição}[section]
\newtheorem{teo}{Teorema}
\newtheorem{lema}{Lema}

\theoremstyle{definition}
\newtheorem{definicao}{Definição}[section]

\theoremstyle{remark}
\newtheorem{obs}{Observação}[section]
\newtheorem{exemp}{Exemplo}[section]

\begin{document}
\section{Teoria de conjuntos}
\begin{prop}
	$$(A-B)\cup B =  A\cup B$$
\end{prop}
\begin{proof}
	\begin{align*}
		x\in (A-B)\cup B &\iff \\~\\
		(x\in A \land x \not \in B) \lor x\in B &\iff \\~\\
		(x\in A\lor x\in B)  \land( x \not \in B \lor x\in B) &\iff \\~\\
		(x\in A\lor x\in B)  \land t &\iff \\~\\
		x\in A\lor x\in B &\iff \\~\\
		x \in A \cup B  &\iff \\~\\
	\end{align*}
\end{proof}
\begin{obs}
	Acima, $t$ representa tautologia.  Algo que sempre tem valor lógico verdadeiro.
\end{obs}
\begin{prop}
	$$A\times (B\cup C ) = (A\times B )\cup( A\times C)$$
\end{prop}
\begin{proof}
	\begin{align*}
		(x,y) \in A\times (B\cup C ) & \iff 
		x\in A \land (y\in B\cup C)  \\~\\ &\iff 
		x\in A \land (y\in B\lor y\in  C)  \\~\\& \iff
		(x\in A \land y\in B) \lor (x\in A \land y\in  C)  \\~\\& \iff
		((x,y)\in A\times B) \lor  ((x,y)\in  A\times C) \\~\\ & \iff
		(x,y)\in (A\times B) \cup  (A\times C)
	\end{align*}
\end{proof}
\begin{lema}
	Existe uma bijeção entre $X$ e  $X \times \left\{ a\right\}$ .
\end{lema}
\begin{proof}
	Seja a função $g: X  \to  X \times \left\{a\right\}$, dada por $g(x) = (x,a)$. Temos $g(p) = g(q) \iff (p,a) = (q,a) \iff p = q $, logo $g$ é injetiva. Dado $(x,a) \in X \times \left\{ a\right\}$, temos $x\in X$ e $a\in \left\{a\right\}$. Logo existe $x\in X$ tal que $g(x) = (x,a) $. Portanto $g$ é sobrejetiva. Como $g$ é injetiva e sobrejetiva, temos  $g$  bijetiva.
\end{proof}
\begin{prop}
	Se $f:X\to Y$ e $g:Y\to Z$ são bijeções, então $(g \circ f): X\to Z$ é uma bijeção.
\end{prop}
\begin{proof}
	Temos $g(f(a)) = g(f(b)) \implies f(a) = f(b) \implies a = b$. Logo $g \circ f$ é injetiva.

	Tomando $z \in Z$. Como $g$ é sobrejetiva, existe $y\in Y$ tal que $g(y) = z$. Como f é sobrejetiva, existe $x\in X$ tal que $f(x) = y$. Logo existe $x\in X$ tal que  $g(f(x)) = g(y) = z$. Logo $g\circ f$ é sobrejetiva.
\end{proof}
\begin{prop}
	Seja $f:X \to Y$ uma função sobrejetiva. $f$ admite inversa à direita.
\end{prop}
\begin{proof}
	Para todo $y\in Y$, temos $f^{-1}(y) \neq \emptyset$, logo existe $x_y \in f^{-1}(y)$ tal que $f(x_y) = y$. Defina $g: Y\to X$, que associa $y\to x_y$ (axioma da escolha). Logo temos $f(g(y)) = f(x_y) = y$.
\end{proof}
\begin{prop}
	Se $f:X\to Y$ é uma injeção então $f': X\to f(X)$, definida como $f'(x) = f(x)$, é uma bijeção.
\end{prop}
\begin{proof}
	Seja $y\in f(X)$. Por definição de $f(X)$, existe $x\in X$ tal que $f(x) = y$. Logo $f'$ é sobrejetiva.
	Dados $a,b\in X$ com $f'(a)  = f(a) = f(b) =  f'(b)$. Como $f$ é injetiva, temos $a = b$, logo $f'$ é injetiva.
\end{proof}
\begin{prop}
	\label{propbijecao}
	Se $f: A\cup B \to C$ é uma bijeção, então $f': A \to C-f(B)$, $a\mapsto f(a)$ é uma bijeção.
\end{prop}
\begin{proof}
	Se $a,b \in A\subset A\cup B$, temos $f'(a) = f'(b) \iff f(a) = f(b) \implies a = b$ ($f$ é injetiva). Logo $f'$ é injetiva.

	Tomando $y \in C-f(B)$. Como $f$ é sobrejetiva, existe $x\in A\cup B$ tal que $f(x) = y$. Se $x\in B$, teríamos $f(x) \in f(B)$, logo $f(x) \not \in C-f(b)$ (contradição). Logo devemos ter $x\in A$. Logo existe $x\in A$ tal que $f'(x) = f(x) = y$. Logo $f'$ é sobrejetiva. 
\end{proof}
\begin{prop}
	Se $f: A \to B$ é uma bijeção e $C \subset B$, então $f': f^{-1}(C) \to C$, $x\mapsto f(x)$ é uma bijeção.
\end{prop}
\begin{proof}
	Se $a,b \in f^{-1}(C) \subset A$, temos $f'(a) = f'(b) \iff f(a) = f(b) \implies a = b$ ($f$ é injetiva). Logo $f'$ é injetiva.

	Tomando $y \in C$. Como $f$ é sobrejetiva, existe $x\in A$ tal que $f(x) = y\in C$. Como $f(x)\in C$, temos $x \in f^{-1}(C)$. Logo existe $x\in f^{-1}(X)$ tal que $f(x) = f'(x) =  y$. Logo $f'$ é sobrejetiva.   
\end{proof}

\begin{prop}
	Seja $f:A\to B$ uma função e $X\subset Y \subset B$. Temos $f^{-1}(X) \subset f^{-1}(Y)$.
\end{prop}
\begin{proof}
	Se $x\in f^{-1}(X)$, temos $f(x)\in X$. Como $X\subset Y$, temos $f(x)\in Y$. Portanto  $x\in f^{-1}(Y)$. Como $x\in f^{-1}(X) \implies x\in f^{-1}(Y)$, temos $f^{-1}(X) \subset f^{-1}(Y)$.
\end{proof}
\begin{prop}
	Seja $f:A\to B$ uma função bijetiva e $X,Y \subset B$. Temos $f^{-1}(X) = f^{-1}(Y) \iff X = Y$.
\end{prop}
\begin{proof}
	Se $X = Y$ é direto. Supondo $f^{-1}(X) = f^{-1}(Y)$. Se $x\in X$, existe $a\in A$ tal que $f(a) = x$. Logo  $a\in f^{-1}(X)$. Portanto $a\in f^{-1}(Y)$. Logo $x = f(a) \in Y$.  Temos $ x = f(a) \in X \implies x = f(a) \in Y$. Para $y\in Y$ é análogo. Logo temos $X = Y$.
\end{proof}
\begin{prop}
	Se existe a bijeção $f: \{a\} \to X$, então $X = \{b\}$ para algum $b$.
\end{prop}
\begin{proof}
	Seja $b = f(a) \in X$. Seja $c\in X$. Como $f$ é sobrejetiva, existe $k\in \left\{a\right\}$ tal que $f(k) = c$. Temos obrigatoriamente que $k = a$, logo $b = f(a) = c$. Logo  $X =\left\{b\right\}$.
\end{proof}
\begin{prop}
	Se $f:A\to B$ e $g: C \to D$ são bijeções, então $h : A\times B \to B\times D$, $h(a,c) = (f(a),g(c))$ é uma bijeção.
\end{prop}
\begin{proof}
	Seja $(b,d) \in B\times D$. Como $f$ e $g$ são sobrejetivas, existem $a\in A$ e $c\in C$ tal que $f(a) = b$ e $g(c) = d$. Logo existe $(a,c) \in A\times C$ tal que $h(a,c) = (f(a),g(c)) = (b,d)$. Logo $h$ é sobrejetiva.

	Suponha $h((a,b)) = h((c,d)) \iff (f(a), g(b)) = (f(c), g(d)) \iff f(a) = f(c) \land g(b) = g(d)$. Como $f$ e $g$ são injetivas, temos $f(a) = f(c) \implies a =c $ e $g(b) = g(d) \implies b = d$. 
	Logo $h$ é injetiva. Como $h$ é injetiva e sobrejetiva, temos que  $h$ é bijetiva.

\end{proof}
\section{Conjuntos Finitos e Infinitos}
\subsection{Números naturais}
\paragraph{Temos como conceitos primitivos o conjunto dos naturais, denotado por $\mathbb{N}$, cujos elementos são os números naturais, e uma função $s:\mathbb{N} \to \mathbb{N}$. Para cada $n\in \mathbb{N}$, o número $s(n)$ é o sucessor de $n$.  Temos os axiomas:}
\begin{axioma}
	$s:\mathbb{N} \to \mathbb{N}$ é injetiva.
\end{axioma}
\begin{axioma}
	$\mathbb{N} - s(n) = \left\{ 1 \right\}$. Ou seja, só existe um número natural que não é sucessor de nenhum outro, e ele é denotado por $1$.
\end{axioma}
\begin{prop}
	\label{propsuc}
	Todo natural diferente de $1$ possui um antecessor.
\end{prop}
\begin{proof}
	Seja $n\neq 1$ um número natural. Suponha que não exista $n_0$ natural com $s(n_0) = n$. Logo $n \not \in s\left(\mathbb{N}\right)$. Logo $n\in \mathbb{N} - s(n)$. Mas $\mathbb{N} - s(n) = \left\{ 1 \right\}$. Logo $n = 1$. Contradição. Logo existe  $n_0\in \mathbb{N}$ tal que $s(n_0) = n$.
\end{proof}
\begin{obs}
	Observe que a função $s: \mathbb{N} \to \mathbb{N}\setminus \{1\}$ é injetiva por definição e sobrejetiva pela proposicao \ref{propsuc}, logo é uma bijeção entre um subconjunto dos naturais com os naturais.
\end{obs}
\begin{axioma}[Princípio de indução]
	Se $X\subset \mathbb{N}$ é  um subconjunto tal que: 
	$$
	\begin{cases}
		1 \in X\\~\\
		n \in X \implies s(n) \in X
	\end{cases}
	$$
	Então $\mathbb{N} = X$.
\end{axioma}
\begin{definicao}[Soma]
	Dados $m,n\in \mathbb{N}$, sua soma $m+n$ é definida como:$$ m+n \coloneqq s^n(m).$$ A soma deve obedecer \begin{equation} m+1 = s(m)\end{equation} \begin{equation}m +s(n) = s(m+n)\end{equation}  para todos os $m,n$ naturais.
\end{definicao}
\begin{obs}
	Dedekind prova o "Teorema da Definição por Indução" para garantir que a notação $s^n(m)$ faça sentido.
\end{obs}
\begin{prop}[Associatividade da Soma]
	Para todos $p,m,n \in \mathbb{N}$, temos $m+(n+p) = (m+n) +p$.
\end{prop}
\begin{proof}
	Seja $X = \left\{ p \in \mathbb{N} \: | \forall m,n\in \mathbb{N} \: : \: m+(n+p) = (m+n) + p \right\}$. Da definição de adição, temos pra qualquer $m,n$ que $n+1 = s(n)$, logo $m+(n+1)  = m+s(n) = s(m+n) = (m+n) +1 \implies m+(n+1) = (m+n) +1 $. Logo $1\in X$.  Se $p \in X$, temos $m+(n+p) = (m+n) +p $. Logo \begin{align*} 
		m+(n+s(p)) &= m+ s(n+p) \\~\\
		&= s\left(m+(n+p)\right) \\~\\ 
		&= s\left((m+n)+p\right) \\~\\ 
		&= (m+n) + s(p).
	\end{align*}
	Logo $p \in X \implies s(p) \in X$. Temos que $X =  \mathbb{N}$ pelo princípio de indução. Logo a soma é associativa nos naturais.
\end{proof}
\begin{lema}[Comutatividade da soma com o $1$]
	\label{lema1}
	Para todo $m \in \mathbb{N}$, temos $m+1 = 1 +m$.
\end{lema}
\begin{proof}
	Seja $X = \left\{ m \in \mathbb{N} \: |  m+1 = 1+m \right\}$. Temos $1\in X$, pois $1+1 = 1+1$. Supondo $m\in X$, logo $m+1 = 1+m$. Temos \begin{align*} 
		1+ s(m)  &= s(1+m) \\~ \\
		&= s(m+1) \\~\\
		&= (m+1) +1 \\~\\
		&= s(m)+1
	\end{align*}
	Como $m\in X \implies s(m) \in X$ e $1\in X$, temos $X= \mathbb{N}$.
\end{proof}
\begin{prop}[Comutatividade da soma]
	Para todos $m,n \in \mathbb{N}$, temos $m+n = n +m$.
\end{prop}
\begin{proof}
	Seja $X = \left\{ m \in \mathbb{N} \: | \forall n \in \mathbb{N} \: : \:  m+n = n+m \right\}$. Temos $1\in X$  pelo  Lema \ref{lema1}.
	Supondo $m\in X$, logo $m+n = n+m$ para todo $n\in \mathbb{N}$. Temos \begin{align*}
		n + s(m) &= s(n+m)  \\~\\
		&= s(m+n) \\~\\
		&= (m+n) + 1\\~\\
		&= 1+(m +n )\\~\\
		&= (1+m )+n \\~\\
		&= (m+1 )+n \\~\\
		&= s(m) +n \\~\\
	\end{align*}
	Como $1\in X$ e $m \in X \implies s(m) \in X$, temos $X = \mathbb{N}$ pelo princípio de indução.
\end{proof}
\begin{prop}[Lei do corte]
	Para todos $m,n, p  \in \mathbb{N}$, temos $m+n = m +p \implies n = p $.
\end{prop}
\begin{proof}
	Seja $X = \left\{ m \in \mathbb{N} \: | \forall n\in \mathbb{N}\: \forall p \in \mathbb{N} \: : \:  m+n = m+p\implies n=p \right\}$. Temos $1\in X$  pois $1+n = 1+p \implies n+1 = p+1 \implies s(n) = s(p) \implies n =p$ pela injetividade de $s$.
	Supondo $m\in X$, temos $m+n = m+ p \implies n = p$ para todos $n,p$ naturais. Temos \begin{align*}
		s(m) + n = s(m)+p &\implies \\~\\
		n + s(m)  =  p + s(m) &\implies \\~\\
		s(n + m)  =  s(p + m) &\implies \\~\\
		n+m = p+m &\implies \\~\\
		m+ n = m+p &\implies \\~\\
		n = p.
	\end{align*}
	Logo $s(m) +n = s(m) + p \implies n = p$.
	Como $1\in X$ e $m \in X \implies s(m) \in X$, temos $X = \mathbb{N}$ pelo princípio de indução.
\end{proof}
\begin{lema}[Não existem ciclos nos naturais]
	\label{lema2}
	Para todos $m,p\in \mathbb{N}$, temos $m\neq m+p$.
\end{lema}
\begin{proof}
	Suponha que $m = m+p$ com $m,p\in \mathbb{N}$. Logo $s(m) = s(m+p) \implies m+1 = (m+p) +1 \implies m+1 = m+(p+1) \implies 1 = p+1 \implies s(p) = 1$. Como $1$ não é sucessor de nenhum natural, temos uma contradição. Logo $m\neq m+p$ para todos naturais $m,p$.
\end{proof}
\begin{lema}[Unicidade da Tricotomia]
	\label{lema3}
	Dados dois naturais $m$ e $n$, apenas uma das 3 possibilidades ocorre:
	$$
	\begin{cases}
		m = n \\~\\
		\exists p \in \mathbb{N} \: : \: m = n+p \\~\\
		\exists q \in \mathbb{N} \:  : \:n = m+q 
	\end{cases}
	$$
\end{lema}
\begin{proof}
	Pelo lema \ref{lema2}, se $m=n$, não podemos ter $m = n+p = m+p$ ou $n = m+q = n+q$ para algum $p,q \in\mathbb{N}$. Se $\exists p \in \mathbb{N} \: : \:m = n+p$, não podemos ter $m = n$ pelo lema \ref{lema2} e não podemos ter $\exists q \in \mathbb{N} \: : \:n = m+q$, pois  teríamos $m = n+p = (m+q)+p = m+(q+p) \implies  m = m +(q+p)$, que contradiz o lema \ref{lema2}.
\end{proof}
\begin{prop}[Tricotomia]
	\label{tricotomia}
	Dados dois naturais $m$ e $n$, exatamente uma das 3 possibilidades ocorre:
	$$
	\begin{cases}
		m = n \\~\\
		\exists p \in \mathbb{N} \: : \: m = n+p \\~\\
		\exists q \in \mathbb{N} \:  : \:n = m+q 
	\end{cases}
	$$
\end{prop}
\begin{proof}
	Seja $X = \left\{m \in \mathbb{N} | \forall n \in \mathbb{N} \: : \: (m = n ) \lor (	\exists p \in \mathbb{N} \: : \: m = n+p ) \lor ( 	\exists q \in \mathbb{N} \: : \: n = m+q ) \right\}$, ou seja: o conjunto dos números naturais que satisfazem pelo menos uma das condições da tricotomia para todo $n$.  

	$1\in X$, pois dado $n\in \mathbb{N}$, temos $n = 1$ ou $n\neq 1$. Se $n=1$, temos $m =1 = n$. Se $n\neq 1$, como $\mathbb{N} - s(\mathbb{N}) = \left\{1\right\}$, temos que existe um $n_0\in \mathbb{N}$ tal que $s(n_0) = n$. Logo $n = n_0 +1 \implies \exists q \: : \: n = q+1  = q+m $.
	
	Supondo $m\in X$. Dado $n\in \mathbb{N}$, se $m = n$, temos $s(m) = s(n) = n+1$, logo $\exists p \in\mathbb{N} \: : \: s(n) = n+p$. Se  $\exists p \in\mathbb{N} \: : \: m = n+p$, temos $s(m) = s(n+p) = (n+p+1) = n +s(p)$ , logo $\exists p' \in\mathbb{N} \: : \: s(n) = n+p'$. Se $\exists q \in \mathbb{N} \: : \: n = m+q $ com $q=1$, temos $n = m+1 = s(m)$. Se $\exists q \in \mathbb{N} \: : \: n = m+q $ com $q\neq 1$, existe $q_0\in \mathbb{N}$ tal que $s(q_0) = q$, logo temos $n = m+q = m +s(q_0) = m+(q_0+1) = m+1+q_0 = s(m)+q_0 \implies \exists q' \in \mathbb{N} \: : \: n = s(m)+q' $. 

	Como $1\in X$ e $m\in X \implies s(m) \in X$, temos $X = \mathbb{N}$.  Logo para todo par $m,n\in \mathbb{N}$, pelo menos uma das condições da tricotomia ocorre. Pelo lema \ref{lema3}, apenas uma das possbilidades ocorre. 
	%Melhorar
\end{proof}
\begin{definicao}[$<$]
	$$ m<n \iff \exists p\in \mathbb{N} \: : \: n = m+p$$
	Dados $m,n$ naturais, dizemos que $m$ é menor que $n$ ( $m<n$) quando existe $p\in \mathbb{N}$ tal que $n = m+p$.
\end{definicao}

\begin{prop}
	Temos $1 < n $ para todo $ 1 \neq n \in \mathbb{N}$ .
\end{prop}
\begin{proof}
	Como $n \neq 1$, temos pela proposição \label{propsuc} que $n$ possui um antecessor. Logo existe $n_0$ tal que $s(n_0) =n \implies n  = 1+n_0$. Logo $1<n$.
\end{proof}
\begin{definicao}[$\leq$]
	$$m \leq n \iff (m = n) \lor (m <n)$$
\end{definicao}
\begin{prop}[Transitividade da relação $<$]
	$m< n \land n< p \implies m < p$
\end{prop}

\begin{proof}
	Se $m<n$ e $n<p$, temos $n = m+q$ e $p = n+r$ para algum par $q,r\in \mathbb{N}$. Logo $p = n +r = (m +q) +r =m +(q+r)$. Logo $m<p$.
\end{proof}
\begin{prop}[Tricotomia da relação $<$]
	Dados $m,n\in \mathbb{N}$, exatamente uma das afirmações ocorre: $m=n$, ou $m<n$, ou $n<m$.
\end{prop}
\begin{proof}
	Segue diretamente da proposição \ref{tricotomia}.
\end{proof}
\begin{prop}
	$$ p\leq q \land q \leq p \iff  p = q$$
\end{prop}
\begin{proof}
	Supondo $p = q$, temos $p\leq q $ e $q \leq p$. 

	Supondo $p\leq q \land q\leq p$. Se $p=q$, acabou a demonstração. Supondo $p\neq q$. Logo devemos ter $p <q $ e $q <p$ (contradição). Logo devemos ter $p=q$.

\end{proof}
\begin{prop}
	Dados $m,n,p$ naturais, temos $$m +p < n+p \implies m <n.$$
\end{prop}
\begin{proof}
	Temos $m+p < n+p \implies \exists q \in \mathbb{N} \: : \: n+p = (m+p) + q \implies  \exists q \in \mathbb{N} \: : \: n = m + q \implies m<n$.
\end{proof}
\begin{lema}
	\label{lemaDesigualdade}
	$$ m < n+1 \iff m \leq n$$
\end{lema}
\begin{proof}
	Supondo $m<n+1$. Logo existe $q\in \mathbb{N}$ tal que $n+1 = m+q$. Se $q = 1$, temos $n+1  = m+1 \implies n = m \implies m \leq n$. Se $q\neq 1$, existe $q_0$ tal que $s(q_0) = q$. Logo $n+1 = m+s(q_0) = m+q_0+1\implies n = m+q_0 \implies m< n \implies m \leq n$. 

	Se $m \leq n$, temos $m\leq n < n+1 \implies m<n+1$.
\end{proof}
\begin{definicao}[Multiplicação]
	Para todo $m\in \mathbb{N}$, seja $f_m : \mathbb{N} \to \mathbb{N}$ que associa cada $p\in \mathbb{N}$ a $f_m(p) = m+p$.
	Dados $m,n \in \mathbb{N}$, o produto entre naturais satisfaz $m\cdot 1 = m$ e  $ m\cdot (n+1) = (f_m)^n(m)$ .
\end{definicao}
\begin{lema}[ Distributiva do sucessor]
	$$m\cdot(n+1) = mn +m$$
\end{lema}
\begin{proof}
	\label{lemaDistributiva1}
	Se $n = 1$, temos  $m\cdot(1+1) = (f_m)^1(m) = f_m(m) = m+m = m\cdot 1 +m$. Se $n\neq 1$, existe $n_0\in \mathbb{N}$ tal que $s(n_0) = n$. Logo temos $m\cdot( n +1) = (f_m)^{n}(m) = (f_m)^{s(n_0)}(m) =   f_m( (f_m)^{n_0} (m)) = f_m( m(n_0+1)) = f_m(m \cdot n ) =  mn+m$.
\end{proof}
\begin{prop}[ Distributiva à esquerda]
	$$m\cdot(n+p) = mn +mp$$
\end{prop}
\begin{proof}
	Seja $X = \left\{p \in \mathbb{N} | \forall m,n \in \mathbb{N} \: : \:  n\cdot(m+p) = nm + np\right\}$. Temos $1\in X$ pelo lema \ref{lemaDistributiva1}. Supondo $p\in X$. Temos \begin{align*}
		n\cdot(m+s(p)) &= n\cdot ((m+p)+1)  \\~\\
		&= n\cdot(m+p) + n\\~\\
		&= nm+np+n\\~\\
		&=nm+n(p+1)\\~\\
		&=nm+n\cdot s(p)
	\end{align*}
	Como $p\in X\implies s(p) \in X $ e $1\in X$, temos $X = \mathbb{N}$.
\end{proof}
\begin{prop}[ Distributiva à direita]
	$$(m+n)\cdot p = mp +np$$
\end{prop}
\begin{proof}
	Seja $X = \left\{p \in \mathbb{N} | \forall m,n \in \mathbb{N} \: : \:  (m+n)\cdot p = mp + np\right\}$. Temos $1\in X$, pos $(m+n)\cdot 1 = m+n = m\cdot 1 + n \cdot 1$ . Supondo $p\in X$. Temos \begin{align*}
		(m+n)\cdot s(p) &= (m+n)\cdot(p+1)  \\~\\
		&= (m+n)\cdot p + (m+n)\\~\\
		&= mp+np+m+n\\~\\
		&=mp+m+ np+n\\~\\
		&=m(p+1)+ n(p+1)\\~\\
		&=m\cdot s(p) + n\cdot s(p)
	\end{align*}
	Como $p\in X\implies s(p) \in X $ e $1\in X$, temos $X = \mathbb{N}$.
\end{proof}
\begin{prop}[Associatividade]
	$$m\cdot(n\cdot p) = (m\cdot n ) \cdot p$$
\end{prop}
\begin{proof}
	Seja $X = \left\{p \in \mathbb{N} | \forall m,n \in \mathbb{N} \: : \:  m\cdot( n\cdot p) = (m\cdot n )\cdot p \right\}$. Temos $m\cdot (n\cdot1) = m\cdot n = (m\cdot n ) \cdot 1$, logo $1\in X$. \\ ~\\ Supondo $p\in X$. Temos 
	 \begin{align*}
		 m\cdot (n\cdot s(p)) &= m\cdot(n\cdot ( p+1) )  \\~\\
		 &= m\cdot ( n\cdot p + n) \\~\\
		 &= m\cdot ( n\cdot p) + m\cdot n \\~\\
		 &= (m\cdot  n)\cdot p + (m\cdot n) \\~\\
		 &= (m\cdot  n)\cdot (p+1) \\~\\
		 &= (m\cdot  n)\cdot s(p) \\~\\
	\end{align*}
	Como $p\in X\implies s(p) \in X $ e $1\in X$, temos $X = \mathbb{N}$.
\end{proof}
\begin{lema}[Comutatividade com $1$]
	\label{comutatividade1}
	$$m\cdot 1 = 1 \cdot m $$
\end{lema}
\begin{proof}
	Seja $X = \left\{m \in \mathbb{N} |  m\cdot 1 = 1\cdot m  \right\}$. Temos $1\cdot 1 = 1\cdot 1  $, logo $1\in X$. Supondo $m\in X$. Temos 
	 \begin{align*}
		 s(m)\cdot 1 &= (m+1)\cdot 1   \\~\\
		 &= m+1  \\~\\
		 &= m\cdot 1 + 1\cdot1  \\~\\
		 &= 1\cdot m +1\cdot 1   \\~\\
		 &=1\cdot (m+1)   \\~\\
		 &=1\cdot s(m)   \\~\\
	\end{align*}
	Como $m\in X\implies s(m) \in X $ e $1\in X$, temos $X = \mathbb{N}$.
\end{proof}
\begin{prop}[Comutatividade]
	$$m\cdot n = n \cdot m $$
\end{prop}
\begin{proof}
	Seja $X = \left\{n \in \mathbb{N} | \forall m \in \mathbb{N} \: : \:  m\cdot n = n\cdot m  \right\}$. Temos $1\in X$ pelo lema \ref{comutatividade1}. Supondo $n\in X$. Temos 
	 \begin{align*}
		 m\cdot s(n) &= m\cdot(n +1 )  \\~\\
		 &=mn + m\cdot1 \\~\\
		 &=nm + 1\cdot m \\~\\
		 &=(n+1)\cdot m  \\~\\
		 &=s(n)\cdot m \\~\\
	\end{align*}
	Como $p\in X\implies s(p) \in X $ e $1\in X$, temos $X = \mathbb{N}$.
\end{proof}
\begin{prop}[Monotonicidade]
	$$ m< n \implies  mp < np $$
\end{prop}
\begin{proof}
	Supondo $m<n$. Logo $n = m+q$ com $q\in \mathbb{N}$. Logo $np = (m+q)p = mp +qp$. Como $qp\in \mathbb{N}$, temos $mp < np$.
\end{proof}
\begin{prop}[Lei do cancelamento]
	$$ mp < np \implies  m < n $$
\end{prop}
\begin{proof}
	Supondo $mp < np$. Pela tricotomia, temos  $n<m$, $m=n$, ou $m < n$. Se $n<m$, temos $np < mp$ (contradição). Se $m = n$, temos $mp = np$ (contradição). Logo devemos ter $m <n$.

\end{proof}
\begin{definicao}[Elemento Mínimo]
	Dado $X\subset \mathbb{N}$, dizemo que $p\in X$ é o menor elemento (ou elemento mínimo) de $X$ se $\forall n\in X \: : \: p\leq n$.
\end{definicao}
\begin{obs}
	Como $\forall n\in \mathbb{N} \: : \: 1 \leq n$, temos que $1\in X$ implica $1$ menor elemento de $X$.
\end{obs}
\begin{prop}
	O elemento mínimo de um conjunto $X\subset \mathbb{N}$, quando existir, é unico.
\end{prop}
\begin{proof}
	Suponha que dado um conjunto $X\subset\mathbb{N}$,  existam $p,q\in X$ elementos mínimos. Logo $p\leq q $ e $q \leq p$. Logo $p=q$.
\end{proof}
\begin{definicao}[Maior elemento]
	Dado $X\subset \mathbb{N}$, dizemo que $p\in X$ é o maior elemento (ou elemento máximo) de $X$ se $\forall n\in X \: : \: p\geq n$.
\end{definicao}
\begin{prop}
	Os naturais não possuem maior elemento.
\end{prop}
\begin{proof}
	Suponha que $x\in \mathbb{N}$ seja o maior elemento de $\mathbb{N}$. Teríamos $s(x)\in \mathbb{N}$ e $x < s(x)$ (contradição). Logo os naturais não possuem maior elemento. 
\end{proof}
\begin{prop}
	O elemento máximo de um conjunto $X\subset \mathbb{N}$, quando existir, é unico.
\end{prop}
\begin{proof}
	Exercício.
\end{proof}
\begin{definicao}[$I_n$]
	$$I_n := \left\{ x\in \mathbb{N} \: | \:  x\leq n \right\}$$
\end{definicao}
\begin{lema}
	$$I_{n+1} = I_n \cup \left\{n+1\right\}$$
\end{lema}
\begin{proof}
	\begin{align*}
		x \in I_{n+1} &\iff \\
		x\leq n+1 &\iff \\
		x < n+1 \lor x = n+1 &\iff \\
		x \leq n \lor x = n+1 & \iff \\
		x\in I_n \lor x \in \left\{n+1\right\} & \iff \\
		x \in I_n \cup \left\{n+1\right\} \\
	\end{align*}
\end{proof}
\begin{teo}[Princípio da boa Ordenação]
	Todo subconjunto $A\neq \varnothing $ dos naturais admite menor elemento.
\end{teo}
\begin{proof}
	Dado $A\subset \mathbb{N}$ não vazio. Se $1\in A$, temos $1$ menor elemento.  

	Supondo $1\not \in A$. Logo $1\in \mathbb{N} - A$.  Seja $X = \left\{ x \in \mathbb{N} \: | \:  I_n \subset \mathbb{N} - A \right\}$.   Como $1\in \mathbb{N} - A$, temos $I_1 = \left\{1\right\} \subset \mathbb{N} - A$, logo $1 \in X$.   Como  $A$ é não vazio, existe $a\in A$. Logo $a\not \in \mathbb{N} -A$.  Temos $a\leq a \implies a\in I_a$. Logo $I_a \not \subset \mathbb{N} - A$.   Logo $a \not \in X$. Temos $1 \in X$ e $X\neq \mathbb{N}$, logo o axioma da indução deve falhar. Logo deve existir $n\in X$ com $n+1 = s(n)\not \in X$. 

	Afirmo que $n+1$ é o menor elemento de $A$.  Como $n\in X$, temos $I_n \subset \mathbb{N} - A$, logo $x \leq n \implies x \in \mathbb{N} - A$. Como $n+1\not \in X$, temos $I_{n+1} \not \subset \mathbb{N} - A$.  Logo existe um $m \in I_{n+1}$ com $m\not \in \mathbb{N} - A\implies m  \in A$. Observe que $m\in I_{n+1} \implies m \leq n+1 \implies m = n+1 \lor m<n+1$. Se $m < n+1$, temos pelo Lema \ref{lemaDesigualdade} que $m\leq n$, que implica $m\in I_{n}$, logo $m\in \mathbb{N} - A$ (contradição). Logo devemos ter $m = n+1$. Temos portanto que $n+1\in A$. 


	Suponha que exista $p \in A$ tal que $p < n+1$.  Teríamos $p\leq n \implies p \in I_n \implies p \in \mathbb{N} -A \implies p \not \in A$. Contradição. Logo temos $n+1\leq p$ para todo $p\in A$. Logo $n+1$ é o menor elemento de $A$.
\end{proof}
\begin{teo}[Indução completa]
	Seja $X \subset \mathbb{N}$ tal que $( \forall m \in \mathbb{N} \: : \: m < n \implies m\in X) \implies n \in X $. Então $X = \mathbb{N}$
\end{teo}
\begin{proof}
	Temos $1 \in X$, pois $1\not \in X$ implicaria na existência de um $m < 1$ com $m\not \in X$.    Supondo $X\neq \mathbb{N}$ e $A = \mathbb{N} - X$. Como $X\neq \mathbb{N}$, temos $A \neq \emptyset$. Logo $A$ possui um menor elemento $a\in A$.  Se $p\in \mathbb{N}$ com $p < a $, então $p \not \in A$, logo $p\in X$. Como $ \forall p \in \mathbb{N} \: : \: p < a \implies p \in X$, temos $a \in X$. Contradição. Logo $A$ é vazio. Logo $X = \mathbb{N}$.
\end{proof}


\section{Conjuntos Finitos e Infinitos}

\begin{definicao}[ Conjuntos finitos]
	Um conjunto $X$ é finito quando for vazio ou quando existir para algum $n\in \mathbb{N}$ uma bijeção $\phi \: :\: I_n \to X$
\end{definicao}
\begin{definicao}[Tamanho de um conjunto]
	Dado um conjunto finito. Dizemos que ele tem zero elementos se for vazio e que ele tem $n$ elementos se tiver bijeção com $I_n$.
\end{definicao}
\begin{obs}
	O conjunto $I_n$ é finito e possui $n$ elementos. 
\end{obs}
\begin{obs}
	Denota-se $|A|$ como o tamanho do conjunto $A$.
\end{obs}
\begin{prop}
	Se $f:X\to Y$ é uma bijeção, então $X$ é finito se, e somente se, $Y$ for finito.
\end{prop}
\begin{proof}
	Se $X$ for finito, então existe um bijeção $\phi \: : \: I_n \to X$.  A composição $(\phi \circ f) \: : \: I_n \to Y$ é uma bijeção, logo $Y$ é finito. O caso $Y$ finito é análogo.
\end{proof}
\begin{teo}
	Seja $A\subset I_n$ não vazio. Se exite uma bijeção $f: I_n \to A$, então $A = I_n$.
	\label{TeoremaSubFinito}
\end{teo}
\begin{proof}
	Seja $X = \left\{n \in \mathbb{N} \: | \: \forall A \subset I_n \: : \: (\text{Existe uma bijeção } f:  I_n \to A ) \implies A = I_n \right\}$. Temos $1\in X$, pois  $I_1 = \left\{1\right\}$ e $A\subset I_1 \implies A = \left\{1\right\}  = I_1$. Supondo $n\in X$.  Seja $A\subset I_{n+1}$ com uma bijeção $f:I_{n+1} \to A$. Restringindo $f$ a $I_n$, obtemos $f': I_n \to A-\left\{f(n+1)\right\}$, que é uma bijeção pela proposição \ref{propbijecao}.

	Se $A - \left\{f(n+1) \right\} \subset I_n$, temos por $n\in X$ que $A-\left\{f(n+1)\right\} = I_n$. Como o contra-domínio de $f$ é $A$ e $A\subset I_{n+1}$, temos que $f(n+1)\in A \implies f(n+1) \in I_{n+1} \implies f(n+1) \in I_n \lor f(n+1)\in \left\{n+1\right\} $. Se $f(n+1) \in I_n$, temos $f(n+1)\not \in A-\left\{f(n+1)\right\}$, logo $A - \left\{ f(n+1) \right\} \neq I_n$ (contradição). Logo temos $f(n+1) = n+1$. Logo $ f(n+1) = n+1 \in A$. Como $ A - \left\{ n+1\right\} = A -\left\{ f(n+1) \right\}  = I_n$, temos $ (A - \left\{ n+1\right\}) \cup \left\{n+1\right\} =   I_n\cup\left\{n+1\right\} \implies A\cup \left\{n+1\right\} = I_{n+1} \implies A = I_{n+1}$. Logo temos $A = I_{n+1}$.


	Se $A - \left\{f(n+1)\right\} \not \subset I_n$. Logo existe $a\in A$ tal que $a\not \in I_n$ e $a\neq f(n+1)$. Mas $A\subset I_{n+1}$. Logo $a\in I_{n+1} = I_n \cup \left\{n+1\right\}$. Logo devemos ter $a  =n+1$. Como $f$ é sobrejetiva, existe $m\in I_{n+1}$ tal que $f(m) = n+1$. Definindo a função $g \: : \: I_{n+1} \to A$, como $g(x) = \begin{cases}
		f(x), &x\neq f(n+1) \land x \neq n+1\\
		n+1, &x = n+1 \\
		f(n+1), &x = m
	\end{cases}$. Temos $g$ uma bijeção. Logo a restrição $g': I_n \to A-\left\{g(n+1) \right\}$ é uma bijeção com $A - \left\{g(n+1) \right\} \subset I_n$ . Portanto temos $A - \left\{g(n+1)\right\} = I_n$ com $A = I_{n+1}$. 
	%melhorar
\end{proof}
\begin{prop}
	Se existe uma bijeção $f: I_n \to I_m$, então $I_m = I_n$.
\end{prop}
\begin{proof}
	Se $m\leq n$, então existe uma bijeção $f:I_n \to I_m$ com $I_m \subset I_n$. Logo pelo teorema anterior, temos $I_m = I_n$. Se $n > m$, temos a bijeção $f^{-1}: I_m \to I_n$ com $I_n \subset I_m$. Logo pelo teorema anterior $I_m = I_n$.
\end{proof}
\begin{prop}
	Não existe uma bijeção $f:X\to Y$ entre um conjunto finito $X$ e uma parte própia $Y \subset X$.
\end{prop}
\begin{proof}
	Como $X$ é finito, existe uma bijeção $g: I_n \to X$. Suponha que exista uma bijeção $f:X\to Y$. Como $Y$ é parte própria, existe um $x\in X -Y$. Tome $A = g^{-1}(Y)\subset g^{-1}(X) = I_n$. Temos $g^{-1}(x) \not \in A$, logo $A$ é uma parte própria de $I_n$. Queremos achar uma bijeção $h: I_n \to A$. Restringindo $g$ a $A$, obtendo a bijeção $g':A\to Y$. Definindo a bijeção $h = (g') \circ f \circ g : I_n \to A$. Pelo teorema \ref{TeoremaSubFinito}, temos que $A = I_n$. Uma contradição, pois $A$ é parte própria de $I_n$. Logo não existe bijeção entre um conjunto finito $X$ e uma parte própria $Y\subset X$.
\end{proof}
\begin{lema}
	Todo subconjunto $A$ de $I_n$ é finito e temos $|A| \leq n$ 
\end{lema}
\begin{proof}
	Seja $X = \left\{ n\in \mathbb{N} \: | \:  A \subset I_n \implies A \text{ finito } \land |A| \leq n \right\}$. Temos $1 \in X$, pois os subconjuntos de $I_1 = \left\{1\right\}$ são $\left\{\right\} $ e $\left\{1\right\} = I_1$, ambos finitos. 

	Suponha $n \in X$.  Seja $A \subset I_{n+1} = I_n\cup \left\{n+1\right\}$.  Se $n+1\not \in A$, então temos $A \subset I_n$. Pela hipótese de indução, temos $A$ finito e $|A| \leq n < n+1$. 

	Supondo $n+1\in A$. Se $A = \left\{n+1\right\}$, temos $A$ finito e $|A| = 1 \leq n$.  Supondo $A \neq \left\{n+1\right\}$, temos $B = A-\left\{n+1\right\} \neq \emptyset$ e $B\subset I_n$. Logo $B$ é finito e temos $ k = |B| \leq n$. Como $B$ é finito, existe a bijeção $f: I_k \to B$. Definindo a bijeção $f': I_{k+1} \to A$ pondo $f'(x) = f(x)$ para $x\in I_n$ e $f(k+1) = n+1$.  Logo $A$ é finito e temos $|A| = k+1 \leq n+1$.


\end{proof}
\begin{lema}
	Seja $A\subset I_n$. Temos $|A| = n \iff A = I_n$.
\end{lema}
\begin{proof}
	Se $|A| = n$, existe a bijeção $f: I_n \to A$,com $A \subset I_n$, logo $A = I_n$.
\end{proof}
\begin{teo}
	Todo subconjunto $Y$ de um conjunto finito $X$ é finito e $|Y| \leq |X|$, com $|Y| = |X| \iff X = Y$.
\end{teo}
\begin{proof}
	Se $X$ é finito, existe uma bijeção $f : I_n \to X$. Seja $A = f^{-1}(Y) \subset I_n$ e seja a bijeção $f': A \to Y$ a restrição de $f$ a $A$.  Como $A\subset I_n$, temos $A$ finito e $|A| \leq n$. Logo $Y$ é finito e  $|Y| = |A| \leq n$.  Temos $|Y| = |A| = n  = |X| \iff |A| = I_n$. Logo $f^{-1}(Y) = I_n = f^{-1}(X)$. Logo $X = Y$.
\end{proof}
\begin{prop}
	Seja $f:X\to Y$ uma função injetiva. Se $Y$ é finito, então $X$ é finito e $|X|\leq |Y|$.
\end{prop}
\begin{proof}
	Como existe a injeção $f: X\to Y$, temos a bijeção $f': X \to f(X)$,  com $f(X)\subset Y$. Como $Y$ é finito, temos $f(X)$ finito e  $|f(X)| \leq Y$. Como existe a bijeção $f':X\to f(X)$, temos $|X| = |f(X)| \leq Y$.
\end{proof}
\begin{prop}
	Seja $f:X\to Y$ uma função sobrejetiva. Se $X$ é finito, então $Y$ é finito e $|Y|\leq |X|$.
\end{prop}
\begin{proof}
	Como $f$ é sobrejetiva, ela admite inversa à direita. Seja $g: Y\to X$ a inversa à direita de $f$.  Se $g(y) = g(y')$, temos $f(g(y)) = f(g(y'))$, logo $y = y'$. Logo $g$ é injetiva. Pela proposição anterior, temos $Y$ finito com $|Y| \leq |X|$. 
\end{proof}
\begin{definicao}[Conjunto infinito]
	Um conjunto é infinito quando não for finito.
\end{definicao}
\begin{obs}
	A função sucessor com o contradomínio reduzido é uma bijeção entre uma parte dos naturais com os naturais:
	$$s: \mathbb{N} \to \mathbb{N} - \left\{1\right\}$$
	Logo os naturais são infinitos.
\end{obs}
\begin{definicao}[Conjunto limitado]
	Um conjunto $X\subset \mathbb{N}$ é limitado quando existe $p\in \mathbb{N}$ tal que $\forall n\in X \: : \: n \leq p$.
\end{definicao}

%
%\begin{definicao}[Soma finita]
	%Seja $A$ um anel, $n\in \mathbb{N}$ e $x_1,x_2,\cdots, x_n \in A$. Definimos indutivamente a soma de $x_1,\cdots, x_n$, como 
%\end{definicao}

\begin{teo}
	Seja $X\subset \mathbb{N}$ não vazio. As seguintes afirmações são equivalentes:
	\begin{itemize}
		\item $X$ é finito.
		\item $X$ é limitado.
		\item $X$ possui maior elemento.
	\end{itemize}
\end{teo}
\begin{proof}
	$\text{(a) } \implies \text{ (b)}$ \\

	Seja $A = \left\{ n\in \mathbb{N} \: | \:  |X| = n \implies X \text{ limitado } \right\}$. Se $|X| = 1$, temos que $X =\left\{a\right\}$ para algum $a\in \mathbb{N}$. Logo $X$ é limitado pelo $a$, pois $a\leq a$. Supondo $n\in X$.  Seja $|X| = n+1$. Logo existe uma bijeção $f: I_{n+1} \to X$. Tomando a bijeção $f':I_n \to X-\left\{f(n+1)\right\}$. Logo $X-\left\{f(n+1)\right\}$ tem tamanho $n$. Pela hipótese de indução, temos $X - \left\{ f(n+1)\right\}$ limitado por um $p \in \mathbb{N}$, ou seja: $\forall t\in X - \left\{f(n+1) \right\} \: : \: t\leq p $. Se $f(n+1) \leq p$, temos que $p$ limita $X$. Se $p\leq f(n+1)$, temos para todo $t\in X - \left\{ f(n+1) \right\}$ que $t\leq p \leq f(n+1)$ e $f(n+1)\leq f(n+1)$, logo $f(n+1)$ limita $X$.

	Como $1\in A$ e $n \in A \implies n+1\in A$, temos $A = \mathbb{N}$


	$\text{(a) } \implies \text{ (b)}$ [Outra forma] \\
	Seja $X = \left\{x_1,x_2,\cdots x_n \right\}$, defina $a = x_1+x_2+\cdots x_n$. Temos $x\leq a $ para todo $x\in X$, logo $X$ é limitado.
	

	$\text{(b) } \implies \text{ (c)}$ \\

	Como $X$ é limitado, existe um $p\in \mathbb{N}$ tal que $\forall n \in X \: : \: n \leq p$. É natural pensar que o maior elemento será o menor dos "limitadores". Logo seja $A = \left\{ p \in \mathbb{N} \: | \: \forall n \in X \: : \: n\leq p \right\}$. $A$ é não vazio, logo é limitado inferiormente por um $a\in A$. Se $a\in X$, $a$ é o maior elemento de $X$.  Supondo $a\not \in X$. Logo temos para todo $n\in X$ que  $n \leq a$, mas nunca  $n = a$, logo temos $n < a$. Se $a = 1$, temos $n<1$ (contradição) . Se $a \neq 1$, existe $a_0$ tal que $a_0 + 1 = a$. Pelo lema \ref{lemaDesigualdade}, obtemos $n < a_0+1 \implies n \leq a_0$ para todo $n\in X$. Uma contradição, pois $a_0\in A$ com $a_0 < a$ ($a$ é o menor elemento de $A$).  Logo devemos ter $a\in X$. Logo $X$ possui maior elemento.

	$\text{(c) } \implies \text{ (a)}$ \\

	Seja $p\in X$ o maior elemento de $X$. Conjecturo que $|X| \leq p$.  Vamos mostrar que $X \subset I_p$.  Seja $x\in X$. Como $p$ é o maior elemento de $X$, temos $x\leq p$. Como $X\subset \mathbb{N}$, temos $x\in \mathbb{N}$. Como $x\in \mathbb{N}$ e $x\leq p$, temos $x\in I_p$. Como $x\in X\implies x\in I_p$, temos $X\subset I_p$. Logo $X$ é finito e $|X| \leq  p$.
\end{proof}
\begin{teo}
	Sejam $X,Y$ conjuntos finitos disjuntos, então $X\cup Y$ é finito e $|X\cup Y| = |X| + |Y|$.
\end{teo}
\begin{proof}
	Sejam $f_x : I_n \to X$ e $f_y : I_m \to Y$ bijeções.
	Seja $f_{xy}: I_{n+m} \to X\cup Y$ definida como: 

	$$ f_{xy}(p) = \begin{cases} f_x(p), & p\leq n \\ f_y(r), &n <  p \leq n+m    \end{cases} $$
	Se $n <p$, existe $r \in \mathbb{N}$ tal que $p = n+r$. Como $p \leq n+m$, temos $r \leq m$.

	Supondo $f_{xy}(p) = f_{xy}(q)$ com $p\neq q$. Logo $p<q$ ou $q<p$. Supondo sem perda de generalidade que $p<q$. Se $n < q \leq n+m$ e $p \leq n$, temos $f_x(p) = f_y(q)$, mas $X$ e $Y$ são disjuntos, logo  devemos ter ou $p < q \leq n$  ou $n < p < q \leq m+n$. Se $ p < q \leq n$, temos $f_x(p) = f_x(q) \implies p = q$ ($f_x$ injetiva). O caso $ n < p < q \leq m+n$ é analogo. Logo $f_{xy}(p) = f_{xy}(q) \implies p = q$ (contradição). Logo devemos ter  $p = q$. Logo $f_{xy}$ é injetiva.

	Seja $p\in X\cup Y$. Logo $p\in X$ ou $p\in Y$. Supondo $p\in X$. Como $f_x$ é sobrejetiva, existe $n_x \in I_n$ tal que $f_x(n_x) = p$. Como $n_x\leq n$, temos $f_{xy}(n_x) = f_x(n_x) = p$.  Se  $p\in Y$. Como $f_y$ é sobrejetiva, existe $n_y \in I_m$ tal que $f_y(n_y) = p$. Como $n_y\leq m$, temos $ n < n+n_y \leq m $  e $f_{xy}(n+n_y) = f_y(n_y) = p \: (n_y = r)$ . Logo $f_{xy}$ é sobrejetiva.
	
	Logo $f_{xy}$ é bijetiva.

	Logo $X\cup Y$ é finito e tem tamanho $n+m = |X| + |Y|$.


	
\end{proof}
\begin{prop}
	Sejam $X,Y$ conjuntos finitos , então $X\cup Y$ é finito e $|X\cup Y| \leq |X| + |Y|$.
\end{prop}
\begin{proof}
	Sejam $f_x : I_n \to X$ e $f_y : I_m \to Y$ bijeções.
	Seja $f_{xy}: I_{n+m} \to X\cup Y$ definida como: 

	$$ f_{xy}(p) = \begin{cases} f_x(p), & p\leq n \\ f_y(r), &n <  p \leq n+m    \end{cases} $$
	Se $n <p$, existe $r \in \mathbb{N}$ tal que $p = n+r$. Como $p \leq n+m$, temos $r \leq m$.


	Seja $p\in X\cup Y$. Logo $p\in X$ ou $p\in Y$. Supondo $p\in X$. Como $f_x$ é sobrejetiva, existe $n_x \in I_n$ tal que $f_x(n_x) = p$. Como $n_x\leq n$, temos $f_{xy}(n_x) = f_x(n_x) = p$.  Se  $p\in Y$. Como $f_y$ é sobrejetiva, existe $n_y \in I_m$ tal que $f_y(n_y) = p$. Como $n_y\leq m$, temos $ n < n+n_y \leq m $  e $f_{xy}(n+n_y) = f_y(n_y) = p \: (n_y = r)$ . Logo $f_{xy}$ é sobrejetiva.
	

	Logo $X\cup Y$ é finito e $|X| + |Y| \leq |X| + |Y|$.

\end{proof}
\begin{prop}
	Temos para todos $m,n\in \mathbb{N}$ que $I_n\times I_m$ é finito e $|I_n\times I_m| = n\cdot m$.
\end{prop}
\begin{proof}
	Seja $X = \left\{ n \in \mathbb{N} \: | \: \forall m\in \mathbb{N} \: : \: |I_n \times I_m| = n\cdot m \right\}$.  Temos $1\in X$, pois para qualquer $m\in \mathbb{N}$, existe uma bijeção entre $I_m$ e $I_m\times I_1$, logo $I_m\times I_1$ é finito e  $|I_m\times I_1| = |I_m| = m = 1\cdot m$.


	Supondo $n \in X$. Dado $m\in \mathbb{N}$, seja $ I_m \times I_{n+1} = I_m\times \left( I_n \cup \left\{n+1\right\} \right) = (I_m \times I_n) \cup \left( I_m \times \left\{ n+1\right\}\right)$. Temos $(I_m \times I_n)$ finito e $|I_m \times I_n| = m\cdot n$ (hipótese de indução) e $I_m\times \left\{n+1\right\}$ finito com  $|I_m\times \left\{n+1\right\}| = m$. Logo $| I_m \times I_{n+1}  |=| (I_m \times I_n) \cup \left( I_m \times \left\{ n+1\right\}\right)| = mn + m = m\cdot(n+1)$.

	Como $1\in X$ e $n\in X \implies n+1\in X$, temos $X = \mathbb{N}$.
\end{proof}
\begin{prop}
	Sejam $X,Y$ conjuntos finitos , então $X\times  Y$ é finito e $|X\times Y| = |X| \times |Y|$.
\end{prop}
\begin{proof}
	Sejam $f_x : I_n \to X$ e $f_y : I_m \to Y$ bijeções. Logo $g: I_n \times I_m \to X\times Y$, definida por $g(p,q) = (f_x(p), f_y(q))$ é uma bijeção. Logo $|X\times Y| = |I_n\times I_m| = m\cdot n = |X| \times |Y|$.
		
\end{proof}

\end{document}
