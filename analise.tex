\documentclass{article}

\usepackage{amsthm}
\usepackage{amssymb}
\usepackage{mathtools}
\usepackage{amsmath}
\theoremstyle{plain}
\newtheorem{axioma}{Axioma}
\newtheorem{prop}{Proposição}

\theoremstyle{definition}
\newtheorem{definicao}{Definição}[section]

\theoremstyle{remark}
\newtheorem{obs}{Observação}[section]
\newtheorem{exemp}{Exemplo}[section]

\begin{document}
\section{Conjuntos Finitos e Infinitos}
\subsection{Números naturais}
\paragraph{Temos como conceitos primitivos o conjunto dos naturais, denotado por $\mathbb{N}$, cujos elementos são os números naturais, e uma função $s:\mathbb{N} \to \mathbb{N}$. Para cada $n\in \mathbb{N}$, o número $s(n)$ é o sucessor de $n$.  Temos os axiomas:}
\begin{axioma}
	$s:\mathbb{N} \to \mathbb{N}$ é injetiva.
\end{axioma}
\begin{axioma}
	$\mathbb{N} - s(n) = \left\{ 1 \right\}$. Ou seja, só existe um número natural que não é sucessor de nenhum outro, e ele é denotado por $1$.
\end{axioma}
\begin{axioma}[Princípio de indução]
	Se $X\subset \mathbb{N}$ é  um subconjunto tal que: 
	$$
	\begin{cases}
		1 \in X\\~\\
		n \in X \implies s(n) \in X
	\end{cases}
	$$
	Então $\mathbb{N} = X$.
\end{axioma}
\begin{definicao}[Soma]
	Dados $m,n\in \mathbb{N}$, sua soma $m+n$ é definida como:$$ m+n \coloneqq s^n(m).$$ A soma deve obedecer \begin{equation} m+1 = s(m)\end{equation} \begin{equation}m +s(n) = s(m+n)\end{equation}  para todos os $m,n$ naturais.
\end{definicao}
\begin{obs}
	Dedekind prova o "Teorema da Definição por Indução" para garantir que a notação $s^n(m)$ faça sentido.
\end{obs}
\begin{prop}[Associatividade da Soma]
	Para todos $p,m,n \in \mathbb{N}$, temos $m+(n+p) = (m+n) +p$.
\end{prop}
\begin{proof}
	Seja $X = \left\{ p \in \mathbb{N} \: | \forall m,n\in \mathbb{N} \: : \: m+(n+p) = (m+n) + p \right\}$. Da definição de adição, temos pra qualquer $m,n$ que $n+1 = s(n)$, logo $m+(n+1)  = m+s(n) = s(m+n) = (m+n) +1 \implies m+(n+1) = (m+n) +1 $. Logo $1\in X$.  Se $p \in X$, temos $m+(n+p) = (m+n) +p $. Logo \begin{align*} 
		m+(n+s(p)) &= m+ s(n+p) \\~\\
		&= s\left(m+(n+p)\right) \\~\\ 
		&= s\left((m+n)+p\right) \\~\\ 
		&= (m+n) + s(p).
	\end{align*}
	Logo $p \in X \implies s(p) \in X$. Temos que $X =  \mathbb{N}$ pelo princípio de indução. Logo a soma é associativa nos naturais.
\end{proof}
\end{document}
